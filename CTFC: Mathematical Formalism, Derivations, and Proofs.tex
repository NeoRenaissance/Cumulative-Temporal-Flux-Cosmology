\documentclass[12pt, a4paper]{article}
\usepackage[utf8]{inputenc}
\usepackage[T1]{fontenc}
\usepackage{amsmath}
\usepackage{amssymb}
\usepackage{geometry}
\usepackage{siunitx}
\usepackage{physics}

% Geometry settings
\geometry{top=2.5cm, bottom=2.5cm, left=2.5cm, right=2.5cm}

\title{\textbf{Cumulative Temporal Flux Cosmology (CTFC)}\\ \large Mathematical Formalism, Derivations, and Proofs}
\author{Architect: Daniel Golden (@NeoRenaissance)}
\date{December 2025}

\begin{document}

\maketitle

\begin{abstract}
This document contains the rigorous algebraic and tensor formalism for the Theory of Realized Potential. It defines the relationships between Mass, Time, and Geometry using the derived \textbf{Golden Constant} ($\sigma$).
\end{abstract}

\section{Fundamental Principles}

\subsection{The Golden Equivalence (Chronometric Form)}
To align with standard metrological practices, we express the fundamental relationship of CTFC by solving for the \textbf{Accumulated History} ($\mathcal{T}$). This posits that mass is a function of realized temporal density.

\begin{equation}
    \mathcal{T} = \frac{M}{\sigma}
\end{equation}

Where:
\begin{itemize}
    \item $M$ is the Rest Mass (\si{kg})
    \item $\sigma$ is the Temporal Density Coefficient
    \item $\mathcal{T}$ is the Volume-Duration of history (\si{m^3 \cdot s})
\end{itemize}

\subsection{The Master Equation (Realization Operator)}
The collapse of the Quantum Potential ($\Psi$) into Relativistic Reality is governed by the Realization Operator ($\hat{R}$), which introduces the density term $\sigma$ into the metric manifold.

\begin{equation}
    \hat{R}_\sigma \ket{\Psi_{future}} \longrightarrow g_{\mu\nu} + \sigma \int \Psi^*\Psi \, dt + \xi \, T_{\mu\nu}^{viscosity}
\end{equation}

\section{The Constitutive Equations}

\subsection{Dynamic Dark Energy (The Expansion Integral)}
The Cosmological Constant is re-defined not as a static energy density, but as a dynamic pressure exerted by the accumulated volume of the past.

\begin{equation}
    \Lambda(t) = \sigma \int_{0}^{t} V(t') \, dt'
\end{equation}

\subsection{Gravitational Hysteresis (The Sediment Equation)}
Dark Matter is modeled as the gravitational wake left by baryonic mass moving through the temporal fluid. The Effective Mass ($M_{eff}$) is the sum of the visible mass and its history.

\begin{equation}
    M_{eff}(t) = M_{bar}(t) + \xi \int_{-\infty}^{t} M_{bar}(t') \cdot e^{-\lambda(t - t')} \, dt'
\end{equation}

\subsection{Metric Drag (The Gravity Equation)}
Gravitational attraction is re-interpreted as the viscous drag (shear stress) created by the interaction between local mass density ($\rho$) and the temporal flux vector ($U$).

\begin{equation}
    D_{\mu\nu} = -(\sigma \rho_{matter}) \left[ \nabla_{\mu} U_{\nu} + \nabla_{\nu} U_{\mu} - \frac{2}{3}g_{\mu\nu}(\nabla_\lambda U^\lambda) \right]
\end{equation}

\section{Derivation of Constants}

\subsection{The Golden Constant ($\sigma$)}
Derived by equating the observed mass-energy of Dark Energy to the accumulated spacetime history of the universe.

\begin{align*}
    \text{Given } \rho_{\Lambda} &\approx 7 \times 10^{-27} \, \si{kg/m^3} \\
    \text{Given } t_{age} &\approx 4.35 \times 10^{17} \, \si{s} \\
    \sigma &= \frac{M_{\Lambda}}{\mathcal{T}_{total}}
\end{align*}

\begin{equation}
    \sigma \approx 5.0 \times 10^{-44} \, \si{kg \cdot m^{-3} \cdot s^{-1}}
\end{equation}

\subsection{The Sediment Constant ($\xi$)}
Derived from the observed ratio of Dark Matter to Baryonic Matter ($\approx 5:1$).

\begin{equation}
    \xi = \frac{5}{t_{age}} \approx 1.15 \times 10^{-17} \, \si{s^{-1}}
\end{equation}

\subsection{Metric Resonance ($C_{gold}$)}
Derived via dimensional analysis linking the Golden Energy ($E = \sigma \mathcal{T} c^2$) to Quantum Mechanics ($E = hf$).

\begin{equation}
    C_{gold} = \frac{\sigma c^2}{h} \approx 6.78 \times 10^6 \, \si{Hz} \, (6.78 \, \text{MHz})
\end{equation}

\subsection{Chronon Density ($\rho_{grain}$)}
The fundamental density of a single "tick" of time (Planck Time).

\begin{equation}
    \rho_{grain} = \sigma \times t_p \approx 2.7 \times 10^{-87} \, \si{kg/m^3}
\end{equation}

\section{Geometric Correlations ($\phi$)}

The derived constants exhibit high-precision correlations with the Golden Ratio ($\phi \approx 1.618$).

\begin{itemize}
    \item \textbf{Grid Resolution (Planck Length):}
    \begin{equation}
        l_p \approx \phi \times 10^{-35} \, \si{m}
    \end{equation}
    \item \textbf{Resonance Harmonic:}
    \begin{equation}
        C_{gold} \approx \phi^4 \times 10^6 \, \si{Hz}
    \end{equation}
    \item \textbf{Fractal Time Dilation:}
    \begin{equation}
        v_{layer} = \frac{c}{\phi^n}
    \end{equation}
\end{itemize}

\section{Stress Tests \& Compatibility}

\subsection{Schrödinger Compatibility}
Substituting Golden Energy into the Hamiltonian operator yields a compatible oscillation function based on historical pressure:
\begin{equation}
    \frac{\partial \Psi}{\partial t} = -i \left( \frac{\sigma c^2}{\hbar} \mathcal{T} \right) \Psi
\end{equation}

\subsection{The Void Velocity Limit}
In regions where matter density approaches zero ($\rho \to 0$), the Drag Tensor vanishes ($D_{\mu\nu} \to 0$).
\begin{equation}
    \lim_{\rho \to 0} \left( \frac{d\tau}{dt} \right) = \infty
\end{equation}
This implies that proper time in cosmic voids flows at a maximal rate, creating the "Null-Temporal" state.

\end{document}
